
\chapter{Koncepcja biblioteki}\label{r:koncepcja}

  W tym rozdziale zostaną przedstawione główne założenia biblioteki Parallel na tle istnijących już modeli programowania równoległego, dostępnych w języku C++.

\section{Cele biblioteki}

  Tworzeniu biblioteki Parallel przyświecały bardzo konkretne cele, których ideą przewodnią było ułatwienie wykorzystywania obliczeń równoległych w programach.
  Wymienionym poniżej celom było podporządkowane projektowanie API i implementacja biblioteki.

\subsection{Wysoka efektywność}

  Jednym z głównych powodów stosowania zrównoleglania obliczeń jest przyspieszanie ich wykonania. Dlatego sama biblioteka do zrównoleglania powinna działać szybko.
  Niedopuszczalną byłaby sytuacja, gdyby program współbieżny wykonywał się wolniej niż jego sekwencyjny odpowiednik.
  Napisanie biblioteki, która byłaby w ogólności bardzo wydajna jest raczej niemożliwe, natomiast powinna być wysoce efekwyna, jeśli jest używana do celów, do których została zaprojektowana.
  Biblioteka Parallel będzie biblioteką ogólnego zastosowania, przy pomocy, które będzie możliwe prowadzenie dowolnych obliczeń.
  Dlatego, oprócz szybkiego działania mechanizmów wbudowanych w bibliotekę, niezbędne jest pozwolenie programiście na podejmowanie decyzji o podziale obliczeń na takie fragmenty, że ich równoległe wykonanie będzie najszybsze.
  
\subsection{Zwiększenie produktywności programisty}
  Problem z efektywnością programisty w przypadku pisania programów równoległych polega na tym, że takie programy są trude do pisania.
  Dlatego wymagają znacznych nakładów czasowych.
  Zrównoleglenie choćby niewielkiego fragmentu programu wymaga często znacznie więcej czasu niż napisanie jego sekwencyjnego odpowiednika.
  Być może dlatego obliczenia równoległe wykorzystywane są wyłącznie wtedy, gdy już nie ma innego sposobu osiągnięcia niezbędnego minimum wydajności programu.
  Biblioteka Parallel celuje w zmianę tego stanu rzeczy, dzięki wprowadzeniu modelu programowania równoległego, który będzie tak samo intuicyjny jak programowanie sekwencyjne.
  Dzięki czemu napisanie kodu, który działa współbieżnie, będzie prawie tak samo szybkie jak kodu sekwencyjnego, co pozwoliłoby uzyskać programistom szybsze programy przy tej samej produktywności.

\subsection{Czytelność kodu}

  Tym, co najbardziej utrudnia zrozumienie programów współbieżnych jest konieczność zrozumienia zależności pomiędzy odrębnymi równolegle działającymi częściami programu.
  Zazwyczaj te zależnośći dotyczą miejsc w kodzie, które są od siebie stosunkowo odległe.
  Mnogość niejawnych zależności i przeplotów wykonań programu sprawiają, że nawet pozornie proste operacje są trudne do poprawnego zaprogramowania.
  Jednym z bardziej wymownych przykładów popierających to stwierdzenie jest problem implementacji semafora uogólnionego przy pomocy semaforów binarnych \cite{gensem}.
  Stąd celem, który został postawiony przed biblioteką Parallel było ukrycie do takiego stopnia, do jakiego to możliwe, obecności równoległości w kodzie.
  Najważniejsze jest to, że struktura programu napisanego przy pomocy biblioteki Parallel nie powinna istotnie różnić się od struktury programu sekwencyjnego.
  Pozwoli to na uzyskanie kodu, który będzie znacznie łatwiej zrozumieć.

\subsection{Transparencja}

  Biblioteka Parallel powinna udostępniać programiście wgląd w to, w jaki sposób oblczenia równoległe będą prowadzone.
  Dzięki temu programista będzie mógł uwzględnić podczas programowania ograniczenia, które wynikają z konstrukcji biblioteki.
  Między innymi będzie mógł dostosować wielkość zlecanych fragmentów obliczeń (ziarnistość obliczeń), tak aby zmaksymalizować wydajność programu.
  
\subsection{Abstrakcja}

  Abstrakcja ukrywa niepotrzebne szczegóły implementacji przed programistą, co pozwala na zwiększenie jego produktywności.

\subsection{Ograniczenie konieczności korzystania z mechanizmów komunikacji i synchronizacji procesów równoległych}

  Projektowawnie komunikacji i synchronizacji w programach współbieżnych jest czymś, co decyduje o fakcie, że programowanie równoległe jest tak trudnym zadaniem.
  Celem biblioteki Parallel jest zdjęcie w znacznym stopniu tego obciążenia z programisty.
  Komunikacja pomiędzy różnymi wątkami wykonania będzie koordynowana przez bibliotekę.
  Biblioteka nie może wyręczyć jednak programisty we wszystkim, ochrona spójności struktur danych pozostanie w rękach programisty.

\section{Inspiracja}

\section{Model programowania w bibliotekce Parallel}

\section{Inne modele programowania równoległego w języku C++}

\section{Analiza cech biblioteki}