
\author{Paweł Tryfon}

\nralbumu{248444}

\title{Parallel -- biblioteka do równoległego obliczania wyrażeń w języku C++}

\tytulang{A library for parallel expression evaluation in C++}

%kierunek: Matematyka, Informatyka, ...
\kierunek{Informatyka}

% Praca wykonana pod kierunkiem:
% (podać tytuł/stopień imię i nazwisko opiekuna
% Instytut
% ew. Wydział ew. Uczelnia (jeżeli nie MIM UW))
\opiekun{dra Marcina Benke\\
  Zakład Logiki Stosowanej
  }

% miesiąc i~rok:
\date{Maj 2011}

%Podać dziedzinę wg klasyfikacji Socrates-Erasmus:
\dziedzina{ 
%11.0 Matematyka, Informatyka:\\ 
%11.1 Matematyka\\ 
%11.2 Statystyka\\ 
11.3 Informatyka\\ 
%11.4 Sztuczna inteligencja\\ 
%11.5 Nauki aktuarialne\\
%11.9 Inne nauki matematyczne i informatyczne
}

%Klasyfikacja tematyczna wedlug AMS (matematyka) lub ACM (informatyka)
\klasyfikacja{D. Software\\
  D.3. Programming languages\\
  D.3.3. Language Constructs and Features\\
  Subject: Concurrent Programming Constructs}

% Słowa kluczowe:
\keywords{obliczenia równoległe, C++, wielowątkowość, leniwe wyliczanie, programowanie generyczne}

% Tu jest dobre miejsce na Twoje własne makra i~środowiska:
%\newtheorem{defi}{Definicja}[section]

% koniec definicji

\begin{document}
\maketitle

%tu idzie streszczenie na strone poczatkowa
\begin{abstract}
  Obecne architektury pozwalają na przyspieszanie działania programów dzięki ich wykonywaniu jednocześnie na kilku procesorach.
  Jednakże skorzystanie z tej możliwości przedstawia istotną trudność dla programistów, gdyż programy wielowątkowe w swojej strukturze bardzo różnią się od programów sekwencyjnych.
  Projektowanie i implementacja programów wykorzystujących współbieżność jest znacznie bardziej czasochłonna oraz wymaga wyższych kwalifikacji.
  Niniejsza praca podejmuje próbę stworzenia biblioteki, która ułatwiłaby zadanie programowania programów wielowątkowych.  
  Głównym priorytetem byłu umożliwienie programiście zrównoleglania obliczeń w zwięzły, zrozumiały i prosty sposób.    
  Obecnie nie istnieje w języku C++ żadna biblioteka oferująca taką funcjonalność.
  Praca przedstawia model prowadzenia obliczeń równoległych w języku C++ oraz prezentuje proponowaną implementację.
  W pracy zostały szczegółowo opisane problemy, które zostały rozwiązane podczas projektowania biblioteki, 
  takie jak: mechanizm przekazywania wyrażeń do wyliczenia równoległego, sposób prowadzenia równoległych obliczeń, 
  sposób zwracania wyników obliczeń oraz metody zapobiegania problemom związanym z prowadzeniem równoległych obliczeń.
\end{abstract}

\tableofcontents
%\listoffigures
%\listoftables